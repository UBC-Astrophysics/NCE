We propose to build a collaborative platform leveraging existing
Canadian built research systems and international standards for data
interchange shared by several domains of knowledge, including
astrophysics, oceanography, remote sensing, fisheries, emergency
management and robotics.  This platform would allow for the exciting
possibility of combining data and algorithms across these different
domains.  Example use cases include:
\begin{enumerate}
\item A forest fire-fighter consulting her smartphone before
  approaching the fire to see an immersive view of when the fire is at
  this moment from near real-time remote sensing combined with GIS
  data of the terrain and perhaps more importantly where it will be in
  five or ten minutes by harnessing an HLA simulation of the
  propagation of the fire using the most recent data on the fire, wind
  and the forest.
\item The development of a robotic emergency rescue first responder
  equipment through a simulation of the design in the actual terrain
  of interest.  With parallel computation, thousands of potential
  designs could be tested and modified using a genetic algorithm in a
  matter of minutes to find the optimal design.
\item Data products for management of natural resources combined with
  climate change, potentially taking into account the subtle interplay
  between melting ice fields and agriculture on a global scale.
\item A public health researcher correlates medical outcomes in a
  private database against geographical data to discover new trends
  underlying disease, without gaining access to any private data.
\end{enumerate}
