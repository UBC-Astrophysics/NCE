% A network must demonstrate that it has the key individuals to establish, expand and intensify relationships with partners internationally to create strong fruitful collaborations. Provide brief biographies of five key participants, summarizing their relevant accomplishments and roles in the proposed platform. Describe how these individuals will drive the Strategic Plan and undertake the Platform activities in the NCE-IKTP network.


The partnership consists of representatives from UVic, UBC, SFU,
Ocean Networks Canada, the Space Telescope Science
Institute, International Virtual Observatory Association, and industry
partners from IBM, Urthecast and Magnetar Games.  The key individuals
highlighted in this brief LOI are all Canadians with established
extensive international collaborations.  Consolidating efforts through
CHORUS will intensify these relationships with partners
internationally and further expand them to include more knowledge
domains.


IBM will provide access to critical cloud-based infrastructure software.  UrtheCast will provide multispectral remote sensing data and have
strong expertise in GIS.  Magnetar games provide expertise on
immersive user interfaces and augmented reality.  We envision
developing web-based prototypes allowing a non-expert citizen
scientist to not only access the data but to interact with the data as
an expert would, creating new simulations and perhaps more importantly
to combine data from different domains to generate new understanding.

\subsection*{Yvonne Coady} 

Yvonne leads the systems research group at the University of Victoria,
exploring cloud-based application infrastructure and scientific
visualization.  Her group has had impact in optimizations and
efficiencies afforded by new hardware and infrastructure in a wide
range of scientific applications, including the Thirty Meter Telescope
and Near Field Tsunami Detection and Warning Systems.  As a
co-recipiant of the University of Victoria's Knowledge Mobilization
Award, and a co-winner of Johnson \& Johnson's Cognition Challenge, she
has been both locally and internationally recognized for her
participation in Knowledge Translation activites.

Yvonne will manage the network as a whole and as the chief software
architect, she manage the software development of CHORUS NCE Knowledge
Translation Platform.

\subsection*{Jeremy Heyl}
Jeremy is aprofessor at the University of British Columbia and a
Canada Research Chair in Neutron Stars and Black Holes.  His research
has focussed on compact objects, the evolution of stars and galaxies.
His discoveries include the first realistic calculations of the
mergers of spiral galaxies to form elliptical galaxies, the first
theoretical calculation and observational measurement of the evolution
of the distribution of the luminosities of galaxies through cosmic
time and the first measurement of the diffusion of stars through a
globular cluster.  All of these discoveries required the development
of new statistical tools and the analysis of large datasets. He is an
expert on high-performance computation and the statistical analysis of
large datasets.

Jeremy Heyl will engage the core group of stakeholders and grow the
network.  He will gather the contributions from the individual
stakeholders, identifying which databases to include and how to
include them.  He will also design the initial set of analysis tools
available within the platform and will develop the interface in
collaboration with partners at STSci and Magnetar Games.

\subsection*{Tania Lado Insua}

Tania has investigated environmental impacts of oil-spills on mussel
communities using population genetics techniques, and holds a Diploma
of Advanced Studies in Marine Biology and Aquaculture from the
University of Vigo, in collaboration with the University of Puerto
Rico, Mayagüez.  She obtained MS and PhD degrees in Ocean Engineering
from the University of Rhode Island with research focusing on applying
models of sediment physical properties to past and present climate
change. Her most recent research includes international collaborations
on diverse research topics such as environmental impact evaluation for
renewable energies, paleoceanography, physical properties of the
sediment, geohazards monitoring, seafloor observatories and
paleoclimate.

Tania will provide expert advice on how include data from in-situ
real-time sensors, from Ocean Networks Canada in particular, and from
other sources as well.  She will also engage other Earth and ocean
scientists within Canada and beyond.

\subsection*{R\'eka Gustafson}

R\'eka is the medical health officer for Vancouver Coastal Health and a
clinical assistant professor in the School of Public Health at the
University of British Columbia. She works on improving hospital and
community physician practice through good science, excellent knowledge
translation, unrelenting advocacy, and extraordinary persistence. She
has been the voice of reasoned science in CDC projects and programs
across the province and continually links the most thoughtful
reflection on data with the realities of public health practice and
ethics.

R\'eka will advocate within the CHORUS colloboration for the needs of
the public health community and reach out for new stakeholders in this
area.

\subsection*{David Schade}

David manages the Canadian Astronomy Data Centre (CADC) at the NRC
Herzberg Institute of Astrophysics (NRC-HIA) in Victoria. He has
worked on the Hubble Space Telescope Medium Deep Survey project and
the analysis HST imaging of faint galaxies in the Canada-France
Redshift Survey. The CADC is a virtual observatory that archives
astrophysical data from a variety of ground-based and space platforms.
David is an expert on data archiving from heterogeneous sources and
interoperability.  He is also a key player in the Canadian Advanced
Network for Astronomy Research (CANFAR).

Like geographical simulations, theoretical astrophysics consumes
prodigious amounts of HPC resources. And like geographic databases,
observational astronomy is data-intensive and requires infrastructure
that has not been readily available from organizations like Compute
Canada. CANFAR was formed as a partnership of university scientists
with the Canadian Astronomy Data Centre which has a 25 year history of
providing data management services to the university community. CANFAR
is delivering cloud storage and processing services to the
data-intensive astronomy community in partnership with Compute Canada
and with the support of CANARIE.  David will provide key guidance on the
design of the CHORUS platform.


