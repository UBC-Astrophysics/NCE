% A network must demonstrate that it has the key individuals to establish, expand and intensify relationships with partners internationally to create strong fruitful collaborations. Provide brief biographies of five key participants, summarizing their relevant accomplishments and roles in the proposed platform. Describe how these individuals will drive the Strategic Plan and undertake the Platform activities in the NCE-IKTP network.


The partnership consists of representatives from UVic, UBC, SFU,
Compute Canada, Ocean Networks Canada, the Space Telescope Science
Institute, International Virtual Observatory Association, and industry
partners from Urthecast and Magnetar Games.  The key individuals
highlighted in this brief LOI are all Canadians with established
extensive international collaborations.  Consolidating efforts through
CHORUS will intensify these relationships with partners
internationally and further expand them to include more knowledge
domains.

UrtheCast will provide multispectral remote sensing data and have
strong expertise in GIS.  Magnetar games provide expertise on
immersive user interfaces and augmented reality.  We envision
developing web-based prototypes allowing a non-expert citizen
scientist to not only access the data but to interact with the data as
an expert would, creating new simulations and perhaps more importantly
to combine data from different domains to generate new understanding.

\subsection*{Yvonne Coady} 

Yvonne leads the systems research group at the University of Victoria,
exploring cloud-based application infrastructure and scientific
visualization.  Her group has had impact in optimizations and
efficiencies afforded by new hardware and infrastructure in a wide
range of scientific applications, including the Thirty Meter Telescope
and Near Field Tsunami Detection and Warning Systems.  As a
co-recipiant of the University of Victoria's Knowledge Mobilization
Award, and a co-winner of Johnson \& Johnson's Cognition Challenge, she
has been both locally and internationally recognized for her
participation in Knowledge Translation activites.


\subsection*{Jeremy Heyl}

\subsection*{Tania Lado Insua}

Tania has investigated environmental impacts of oil-spills on mussel
communities using population genetics techniques, and holds a Diploma
of Advanced Studies in Marine Biology and Aquaculture from the
University of Vigo, in collaboration with the University of Puerto
Rico, Mayagüez.  She obtained MS and PhD degrees in Ocean Engineering
from the University of Rhode Island with research focusing on applying
models of sediment physical properties to past and present climate
change. Her most recent research includes international collaborations
on diverse research topics such as environmental impact evaluation for
renewable energies, paleoceanography, physical properties of the
sediment, geohazards monitoring, seafloor observatories and
paleoclimate.


\subsection*{Brian Thom}

Brian’s research focus is on the political, social and cultural
processes that surround Indigenous people's efforts to resolve
Aboriginal title and rights claims and establish self-government.  His
written work explores the interplay of culture, power and colonial
discourses in land claims negotiations, and examines the political and
ontological challenges for Indigenous people engaged with institutions
of the state.  He is principle investigator for the research project
Innovations in Ethnographic Mapping and Indigenous Cartographies,
currently funded by SSHRC and Google. This project grapples with the
practical problem of implementing socially and politically powerful
mapping initiatives which can effectively visualize and communicate
indigenous peoples’ knowledge and experience of the land. Through
cutting-edge immersive, networked, multimedia cartographic systems,
this project offers a response to the critique that the dots and lines
of conventional ethnographic mapping reduces and essentializes
indigenous territoriality and senses of place.


\subsection*{Keith Beckett}

Keith is known as an energetic, highly-productive software, systems
and project engineer, successfully tackling extremely challenging
problems, bringing out-of-the-box thinking to the table and getting
the job done. With a vast experience in leading large and small system
development teams, architecting customer-focused solutions, systems
and software engineering, critical analysis and research.  His
specialties include: Systems Engineering, Software Engineering,
Computer-based Systems, Numerical Analysis and Methods, Image and
Signal Processing, Remote Sensing using both Optical and SAR Sensors,
Modeling and Simulation, High-Performance Computing, and Cloud
Computing.

\subsection*{Duncan Suttles}
