\documentclass[pdftex,12pt]{article}
\usepackage{times} %charter font
\usepackage[OT1, T1]{fontenc} %use TeX encoding then Type 1
\usepackage{graphicx}
\usepackage{supertight}
\usepackage{hyperref}
\usepackage[square]{natbib}
\newcommand{\mn}{{\it MNRAS}}
\newcommand{\mnras}{\mn}
\newcommand{\aj}{{\it AJ}}
\newcommand{\apj}{{\it ApJ}}
\newcommand{\apjl}{{\it ApJL}}
\newcommand{\prl}{{\it PRL}}
\newcommand{\pra}{{\it PRA}}
\newcommand{\prd}{{\it PRD}}
\newcommand{\pre}{{\it PRE}}
\renewcommand{\baselinestretch}{0.97}
\begin{document}
\pagestyle{myheadings}
\markright{Y. Coady (XXXXXX): Global Virtual Observatory}

\section{Vision for the Platform (0.5 page)}

We propose to build a collaborative platform leveraging existing
Canadian built research systems and international standards for data
interchange shared by several domains of knowledge, including
astrophysics, oceanography, remote sensing, fisheries, emergency
management and robotics.  This platform would allow for the exciting
possibility of combining data and algorithms across these different
domains.  Example use cases include:
\begin{enumerate}
\item A forest fire-fighter consulting her smartphone before
  approaching the fire to see an immersive view of when the fire is at
  this moment from near real-time remote sensing combined with GIS
  data of the terrain and perhaps more importantly where it will be in
  five or ten minutes by harnessing an HLA simulation of the
  propagation of the fire using the most recent data on the fire, wind
  and the forest.
\item The development of a robotic emergency rescue first responder
  equipment through a simulation of the design in the actual terrain
  of interest.  With parallel computation, thousands of potential
  designs could be tested and modified using a genetic algorithm in a
  matter of minutes to find the optimal design.
\item Data products for management of natural resources combined with
  climate change, potentially taking into account the subtle interplay
  between melting ice fields and agriculture on a global scale.
\item A public health researcher correlates medical outcomes in a
  private database against geographical data to discover new trends
  underlying disease, without gaining access to any private data.
\end{enumerate}

\section{Expected Impacts and Added Value  (1 page)}


Many Canadian built research tools either involve domain specific
geographical data (e.g. Ocean Networks Canada) or specialized
provisioning of distributed computational resources
(e.g. CANFAR). Currently, to interact with this data or perform
computations, one not only has to be an expert in the particular
sub-discipline but also in each interface of the underlying tool or
the data representation.  We believe these efforts are now mature
enough to compose their collective APIs into one standard interface
available for users from around the world to collaborate through a web
browser.

The key tenet of the proposed platform will be to exploit existing
tools, data sets, and standards to provide an interactive interface
for industry stakeholders and the public.  This will enable
multi-disciplinary and multi-sectoral networks of stakeholders to
create strategic international partnerships. Not only will exchanges
be accelerated, but existing Canadian efforts will be better
internationalized.  Promotion of standards would dramatically increase
access to existing infrastructure and data, foster collaboration among
international stakeholders and ultimately yield new understanding of
our environment, natural resources, more rapid development of robotic
technology and better situational awareness for defense and emergency
management.


\section{Model for Collaboration (1 page)}

\section{Strategic Plan (2 pages)}

gathering of existing tools: SEDRIS, HLA, VO, MAST interface

gathering of data sets and algorithms (new stakeholders)

development of interoperability tools and standard

identification of infrastructure partners (google, amazon, microsoft):
where will the data go, how to get computing, how to put it all together

\section{Proposed Team (2 pages)}

The partnership consists of representatives from UVic, UBC, SFU,
Compute Canada, tOcean Networks Canada, he Space Telescope Science
Institute, International Virtual Observatory Association, and industry
partners from Urthecast and Magnetar Games.  UrtheCast will provide
multispectral remote sensing data and have strong expertise in GIS.
Magnetar games provide expertise on immersive user interfaces and
augmented reality.  We envision developing web-based prototypes
allowing a non-expert citizen scientist to not only access the data
but to interact with the data as an expert would, creating new
simulations and perhaps more importantly to combine data from
different domains to generate new understanding.

\section{ Management and Governance }
\end{document}
