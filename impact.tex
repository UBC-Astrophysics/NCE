
Many Canadian built research tools either involve domain specific
geographical data (e.g. Ocean Networks Canada) or specialized
provisioning of distributed computational resources
(e.g. CANFAR). Currently, to interact with this data or perform
computations, one not only has to be an expert in the particular
sub-discipline but also in each interface of the underlying tool or
the data representation.  We believe these efforts are now mature
enough to compose their collective APIs into one standard interface
available for users from around the world to collaborate through a web
browser.

The key tenet of the proposed platform will be to exploit existing
tools, data sets, and standards to provide an interactive interface
for industry stakeholders and the public.  This will enable
multi-disciplinary and multi-sectoral networks of stakeholders to
create strategic international partnerships. Not only will exchanges
be accelerated, but existing Canadian efforts will be better
internationalized.  Promotion of standards would dramatically increase
access to existing infrastructure and data, foster collaboration among
international stakeholders and ultimately yield new understanding of
our environment, natural resources, more rapid development of robotic
technology and better situational awareness for defense and emergency
management.


