
Many Canadian built research tools either involve domain specific
geographical data (e.g. Ocean Networks Canada) or specialized
provisioning of distributed computational resources
(e.g. CANFAR). Currently, to interact with this data or perform
computations, one not only has to be an expert in the particular
sub-discipline but also in each interface of the underlying tool or
the data representation.  We believe these efforts are now mature
enough to compose their collective APIs into one standard interface
available for users from around the world to collaborate through a web
browser.

The key tenet of the proposed platform will be to exploit existing
tools, data sets, and standards to provide an interactive interface
for industry stakeholders and the public.  This will enable
multi-disciplinary and multi-sectoral networks of stakeholders to
create strategic international partnerships. Not only will exchanges
be accelerated, but existing Canadian efforts will be better
internationalized.  Promotion of standards would dramatically increase
access to existing infrastructure and data, foster collaboration among
international stakeholders and ultimately yield new understanding of
our environment, natural resources, more rapid development of robotic
technology and better situational awareness for defense and emergency
management.

CHORUS will enable partners to deploy new sophisticated information
products take advantage of the new data that will be available in the
near future (such as constellations of high resolution SAR and Optical
satellites).  The enhanced cloud platform will have the capability to
perform big data analytics and generate powerful new information
products based on the analytics results.  Furthermore, we will combine
heterogenous remote sensing and in situ data in a uniform interface
and catalogue which will for the first time allow global calibration
of the remote sensing, dramatically increasing inherent value of the
data and will create an entire new capability of analysing these
heterogenously acquired dataset in tandem.  We envision the technology
to have multiple international market opportunities.

The first is to exploit the powerful new information products that can
be generated from big data analytics combined with the huge data
archive. These opportunities include creating services around these
information products and/or selling the information products to
service companies.  Big data analytics is a rapidly growing market
segment that is growing at a 26.4\% compound annual rate and is
projected to be \$41.5 B annually by 2018 (around the time this project
will be underway and the technology is ready for exploitation). This
dwarfs the global earth observation market which is currently around
\$1.9 B annually and projected to get to \$3B annually by 2020.
Enhancements made to CHORUS will allow competing directly in the big
data analytics market with a powerful cloud based system. Therefore,
the market potential for this technology is very high, and it only
takes a fraction of a percent of market share to see significant
revenues.

Another application of this technology is to take CHORUS and deploy
this entire ecosystem on to a private cloud infrastructure that can be
sold to customers around the world as a stand alone product.  This
targets mainly large government organizations and institutions and
also large companies (e.g., insurance companies) that are interested
in the cloud based computing capability but want to have the system
behind their own firewall so they have complete control of the
security.  The value of this technology is very high to each of these
customers (i.e., easily in the 10's of million dollars) which is what
it would cost for them to develop this capability plus it would cause
them significant delays in acquiring the capability if they developed
it themselves. Therefore, a price of several million dollars per
instance is readily defendable.  Since there are hundreds of
organizations around the world that would be highly interested in this
type of capability, the revenue potential is greater than \$100M.

