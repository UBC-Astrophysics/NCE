By offering an open and collaborative platform, CHORUS will
strengthen, expand and intensify meaningful international
collaborations across a variety of knowledge domains, including
astrophysics, oceanography, remote sensing, fisheries, natural
resources development, energy, agriculture, emergency management,
public health, social services and robotics.  The need for a common
infrastructure, in particular for applications that crosscut both the
knowledge domains and datasets involved, will build links and ensure
shared benefits between national and international institutions alike.
We have successfully assembled leading practioners from both academic
and industry stakeholders involved in extracting information from
existing Earth Observation platforms as well as databases from
observations on the ground.  This initial collaboration will identified
additional potential stakeholders who will assist the software development
tean to address their particular needs.

We will be using several open standards for representation of
geographic information, images and distributed computation.  These
standards are flexible, extensible, well-documented and supported by a
large industry performing both defense and industrial simulation.  New
stakeholders can easily integrate their database in the CHORUS
platform by either providing an API that we will adapt for CHORUS or
using the open HLA standard for distributed computation.  Furthermore,
if a new stakeholder wishes to allow model testing on their data
without divulging the data itself (perhaps due to privacy concerns),
CHORUS would include access to a server that would be restricted to
testing models on the private data.  By distributing the computation
to the data repository, the end user can provide a prospective model
to a mutually trusted server that has access to the private database.
The server would return a single number or a few numbers to quantify
the quality the fit of the model to the data .  
          
