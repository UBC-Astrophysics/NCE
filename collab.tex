By offering an open and collaborative platform, CHORUS will
strengthen, expand and intensify meaningful international
collaborations across a variety of knowledge domains, including
astrophysics, oceanography, remote sensing, fisheries, natural
resources development, energy, agriculture, emergency management,
public health, social services and robotics.  The need for a common
infrastructure, in particular for applications that crosscut both the
knowledge domains and datasets involved, will build links and ensure
shared benefits between national and international institutions alike.
We have successfully assembled leading practioners from both academic
and industry stakeholders involved in extracting information from
existing Earth Observation platforms as well as databases from
observations on the ground.  This initial collaboration will identified
additional potential stakeholders who will assist the software development
tean to address their particular needs.

We will be using several open standards for representation of
geographic information, images and distributed computation.  These
standards are flexible, extensible, well-documented and supported by a
large industry performing both defense and industrial simulation.  New
stakeholders can easily integrate their database in the CHORUS
platform by either providing an API that we will adapt for CHORUS or
using the open HLA standard for distributed computation.  Furthermore,
if a new stakeholder wishes to allow model testing on their data
without divulging the data itself (perhaps due to privacy concerns),
CHORUS would include access to a server that would be restricted to
testing models on the private data.  By distributing the computation
to the data repository, the end user can provide a prospective model
to a mutually trusted server that has access to the private database.
The server would return a single number or a few numbers to quantify
the quality the fit of the model to the data.

The members of the network will meet four times yearly in Vancouver or
Victoria in person, but stakeholders will also collaborate informally
with the software development team at UVic either through in-person
meetings or virtually.  The initial group of stakeholders will
identify additional partners to understand the detailed needs of the
various communities for access and analysis of geographic data (market
discovery).  The team will develop the specifications required of the
platform and in collaboration with the software team create prototypes
of the various data models and database access.  The collaboration
will strive both to identify the particular needs and challenges of
developing the CHORUS platform for knowledge translation and sharing
(market validation), but also to define these needs and challenges by
producing software prototypes to achieve or at least attempt to
achieve the goals (productization).  For this to be most productive,
the key collaboration will be between stakeholders who outline their
needs and the software team that will codify these needs within
software prototypes.  We believe that SEDRIS standard for GIS data and
HLA standard for distributed computation are sufficiently flexible and
extensible to meet the needs of various stakeholders but we will also
explore additional possible data and computational models in
collaboration with our stakeholders at CADC, Urthecast and Magnetar.



          
